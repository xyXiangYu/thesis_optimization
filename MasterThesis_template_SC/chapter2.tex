



\chapter{Rocket car case}
Since the approaches we are going to use in this paper will be demonstrated with the case of rocket car, we decide to describe the rocket car case first. So that, when we are discussing our approaches, we can directly describe how they can be used in solving the rocket car case. The description of the rocket car case is mostly coming from the paper \cite{MatSch22}, with content either verbatim or in a modified form. 

We consider the rocket car case with state constraints, i.e. the one-dimensional movement of a mass point under the influence of some constant acceleration/deceleration, e.g. modeling head-wind or sliding friction, which can accelerate and decelerate in order to reach a desired position. The mass of the car is normalized to 1 unit\footnote{We do not specify the unit on purpose since the actual unit, either one kilogram or meter, does not play a role in the modeling. We are more concerned about the scale.} and the constant acceleration/deceleration enters the model in form of an unknown parameter $p \in \Omega_P \subset \mathbb{R}$ suffering from uncertainty, with the uncertainty set $\Omega_P$ convex and compact. We consider a problem in which the rocket car shall reach a final feasible position and velocity in a minimum time: 



\begin{subequations}
	\begin{align}
		\underset{T, u(\cdot), x(:,p)}{min} \   & \  T \\ 
		s.t.  & \ \ x = (x_1, x_2)   \label{rc_x} \\ 
		& \ \  \dot{x} = T  \begin{pmatrix}  x_2(t;p) \\ u(t)-p   \end{pmatrix}, & \ t \in [0,1],  \label{rc_partial} \\
		& \ \ x(0,p) = 0, \label{rc_t0}\\
		& \ \ x_1(1;p) \geq 10, \label{rc_x1_t1} \\
		& \ \ x_2(t;p) \leq 4, & t \in [0,1], \label{rc_x2_tc} \\
		& \ \ x_2(1;p) \leq 0, \label{rc_x2_t1}  \\
		& \ \ T \geq 0, \\
		& \ \ u(t) \in [-10, 10], & t \in [0,1]. 
	\end{align}
	\label{rc}
\end{subequations}

where $x$ represents the variables of the rocket car, and it has two components $ x = (x_1, x_2)$. The first component $x_1$ is the (time-transformed) position of the rocket car. The second component $x_2$ is (time-transformed) velocity of the rocket car. The condition \ref{rc_t0}, i.e. $x(0,p) = 0$, indicates that at $t=0$, both the position and velocity of the car is $0$. The condition \ref{rc_x1_t1}, i.e. $x_1(1;p) \geq 10$, indicates that the position of the car at $t=1$ must be greater or equal to $10$. The condition \ref{rc_x2_t1}, i.e. $x_2(t;p) \leq 4$, indicates that the velocity of the car is always smaller or equal to 4 across the whole period. The condition \ref{rc_x2_t1}, i.e. $x_2(1;p) \leq 0$, indicates that the velocity of the car at $t=1$ is always smaller or equal to $0$. Here, a negative velocity means that the car is moving in a direction that decreases the position. To make the rocket car case even simpler, we can limit the size of the uncertainty set, as following
\begin{equation}
	p \in \Omega_P = [p_l, p_u] \subset [0,9],
\end{equation}
where $p_l < p_u$, with $p_l$ and $p_u$ the lower and upper boundary of the parameter $p$.  

The decision variable in the problem \ref{rc} is the controllable parameter T, which encodes the process duration of the corresponding problem with free end time. The control function $ u: [0,1] \rightarrow \mathbb{R}$ represents the acceleration/deceleration value, and is dependent on the unknown parameter $p$, as shown in the condition \ref{rc_partial}. The second component of the condition \ref{rc_partial}, i.e. $\dot{x_2} = T (u(t)-p)$, indicates the change in the velocity of the car at time $t$ is subject to the value of $T, u(t)$ and $p$. The first component of the condition \ref{rc_partial}, i.e. $\dot{x_1} = Tx_2(t;p)$, indicates the position of the car at time $t$ is subject to the value of $T$ and the velocity $x_2(t;p)$ at time $t$. The variable $x(t:p)$ is a dependent variable, and is uniquely determined by $T, u(\cdot)$ and $p$. The goal is to minimize $T$ such that the variable $x(t:p)$ satisfies all the conditions in \ref{rc}. 

\section{Theoretical solution to rocket car case}
As explained in Chapter \ref{Chapter1}, we choose the rocket car case for two main reasons, i.e. the easyness of understanding and the existence of theoretical solution, which will be shown in this section. 



%There are three optimization variables in the optimization problem \ref{rc}, i.e. $T, u$ and $x$, and they belong to the following normed space
%\begin{equation}
%	(T, u(\cdot), x(:,p)) \in \mathbb{R} \times \mathbb{L}^\infty([0,1], \mathbb{R}) \times  \mathbb{W}^{1,\infty}([0,1], \mathbb{R}^2)
%\end{equation}
The optimization problem \ref{rc} has a unique global solution, and no further local solution exists. The optimal controllable parameter is given by 
\begin{equation}
	T^\star = T^\star(p) = 2.5 + \frac{40}{100-p^2},
\end{equation}
and the optimal control function $u^\star(\cdot) (= u^\star(\cdot; p))$ by 
\begin{equation}
	u^\star(\cdot) =     \left\{
	\begin{array}{ll}
		10, & for \  0 \leq t <  \frac{4}{(10-p)T^\star}\\
		p  &  for \ \frac{4}{(10-p)T^\star} \leq t < 1- \frac{4}{(10+p)T^\star} \\
		-10  & for \  1- \frac{4}{(10+p)T^\star} \leq t \leq 1 
	\end{array}
	\right.
\end{equation}

In words, we accelerate as strongly as possible (the acceleration value $u^\star(t)=10$) until the velocity $x^\star_2(t;p)=4$, and then keep the velocity $x^\star_2(t;p)$ constant for a certain periofd of time\footnote{The acceleration value cancels out with a inherent deceleration value so that the velocity can stay constant. The inherent deceleration value can be result of a friction or head wind.}, and eventually decelerate as as strongly as possible until the velocity is $x_2(1;p) \leq 0, \label{rc_x2_t1}$. The moment $x_2(1;p)$ reaching the value of $0$ is the moment that we finds the optimal/smallest $T$ that all the conditions are satisfied. 

The proof of the theoretical solution is given in Appendix B of \cite{MatSch22}. Because of the simplicity nature of the rocket car case, we can find the theoretical solution of the nominal/original problem for our case \ref{rc}. But for many real life problems, it is very difficult to find a direct solution to the original problem, and for some cases not feasible, due to the uncertainty in the parameter $p$. That is why in the paper \cite{MatSch22}, a classical (in the form $minmax$) approach and a training (in the form $maxmin$) approach have been discussed, and both approaches will lead to a conservative solution to the original problem. A conservative solution to the CP problem is a acceptable (or desired) result since less risk should be taken regarding the therapy design of CP problem. In the next chapter, we discuss, in details, the classical approaches and training approach for the rocket car problem. After that, we focus on the quasi-Newton and multi shooting approach to the same problem.  





