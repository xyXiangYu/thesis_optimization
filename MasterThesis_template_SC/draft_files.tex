% In the example give above, we have chosen the support function to be a constant value. In practice, the support 
%We have shown a simple problem where numerical solution can be obtained via the mutilple shooting method. In this examplary problem, the 
%The example we show is a 
%$y(\tau_{k+1}; \hat{y}_k) - \hat{y}_{k+1} = 0, \  k = 0, 1, ..., m-1$ and $\hat{y}_{m} - y_f =0$, then stop,
%% $y(\tau_{k+1}; \hat{y}_k) - \hat{y}_{k+1} = 0, \  k = 0, 1, ..., m-1$ and $\hat{y}_{m} - y_f =0$, then stop,

%\usepackage{enumitem}
%\renewcommand{\labelenumii}{\arabic{enumi}.\arabic{enumii}}
%\renewcommand{\labelenumiii}{\arabic{enumi}.\arabic{enumii}.\arabic{enumiii}}
%\renewcommand{\labelenumiv}{\arabic{enumi}.\arabic{enumii}.\arabic{enumiii}.\arabic{enumiv}}

%\usepackage{enumitem}
%\usepackage[shortlabels]{enumitem}

%\usepackage[thmmarks,amsmath]{ntheorem}
%\usepackage{tikz}
%\usepackage[inline]{enumitem}
%\usepackage[thmmarks]{ntheorem}
%\usepackage{cleveref}



%where $p^0$ is a fixed value in the feasible uncertainty set $\Omega_P$, where the parameter $p$ can take value from.In the paper \cite{MatSch22}, multiple methods of solving the parametric optimization problem have been discussed. The main focus (of solving the Cerebral Palsy problem) of the paper \cite{MatSch22}, is the "worst-case treatment planning by bilevel optimal control", i.e. a bilevel optimization problem. The bilevel optimisation method in paper \cite{MatSch22} solves the parametric optimization problems, e.g. the Cerebral Palsy problem, in a conservative way.One method of solving the original CP problem in a conservative way is to transform the problem \ref{ParaMin} into another form. 

%Based on theorem \ref{TH_KKT}, We want to solve equation \ref{eq:GradientLagrange}.
%\begin{align}
%	\label{eq:GradientLagrange}
%	\nabla 	\mathcal{L}(w,\lambda, \mu) = 0.
%\end{align}
%where its derivatives is as follows
%\begin{equation}
%\begin{align}
%	\nabla \mathcal{L}(w,\lambda, \mu) &= 
%	\mtrx{
	%		\nabla_w f(w) +  \nabla_w g(w) \lambda + \nabla_w h(w) \mu  \\
	%		g(w)    \\
	%		h(w)
	%	} 
%\end{align}

%	\nabla^2 \mathcal{L}(w,\lambda, \mu) &=
%\mtrx{
	%	\nabla^2_w f(w) + \nabla^2_w g(w)\lambda  & \nabla_w g(w) & \nabla_w h(w)\\
	%	\nabla_w g(w) ^\top\\
	%	\nabla_w h(w)^\top
	%}


%\end{equation}
% 	\mathcal{L}(w,\lambda, \mu) &= \Phi(w) - \lambda ^\top a(w)-  \mu^\top b(w) \\




%We apply Newton's method and we have to solve for $(w_i,\lambda_i, \mu_i)$ :
%\begin{align}
%	\nabla^2 \mathcal{L}(w_i,\lambda_i, \mu_i) \Delta w +  \nabla \mathcal{L}(w_i,\lambda_i, \mu_i) = 0.
%\end{align}

%Equivalently, the following QP ( Quadratic Programming ) needs to be solved:




% and include the equality
%\begin{equation}
%	\mathcal{L}(x,\lambda, \mu) = f(x) + \lambda^\top g(x) +  \mu^\top h(x) 
%	\label{eq_Lagrangian}
%\end{equation}




%The optimal control problem (OCP) \ref{P2_OPM} involves constraints in the partial differential equation form and in a time horizon $t \in [t_0, t_f]$. Therefore, we can apply multiple shooting method discretize the original OCP of the whole interval into mutiple OCPs in different subintervals, with the constraints enforced at the boundary of subintervals to guarantee the continuity. Together with the matching conditions, we can then aggregate the subproblems and apply quasi Newton method to get the final optimal solution. The numerical solution, with multilpe shooting and quasi Newton method to solve optimal control problem, will be demnostrated with rocket car case in Chapter \ref{Chapter4}.  

%In the next section, we explain the multilpe shooting method first and in the next chapter we then focus on how th




%The constrained optimization \ref{eq:OCP_discret_compact} can form the Lagrangian function
%\begin{equation}
%	 \mathcal{L}(w,\lambda, \mu) = \Phi(w) - \lambda ^\top a(w)-  \mu^\top b(w) 
%\end{equation}


%\definition (Active Constraints and Active Set) An inequality constraint hi(x) ≤ 0 is called active at x∗ 2 Ω
%iff hi(x∗) = 0 and otherwise inactive. The index set A(x∗) ⊂ f1; : : : ; nhg of active inequality constraint indices
%is called the ”active set”




%The Karush–Kuhn–Tucker theorem then states the following.


%% = \frac{x_2(t_k) + x_2(t_{k-1})}{2} (\tau_k -\tau_{k_1})
%%%  \frac{x_2(t_k) + x_2(t_{k-1})}{2} (\tau_k -\tau_{k_1}) 




% t&= \frac{\tau}{T} \in [0,1] \quad \tau \in [0, T]
%\\
%i.e. \ \begin{pmatrix} \dot{x_1} \\ \dot{x_2} \end{pmatrix} &= \begin{pmatrix} \frac{\partial x_1}{\partial t} \\ \frac{\partial x_2}{\partial t} \end{pmatrix} = 
%\begin{pmatrix} \frac{\partial x_1}{\partial \tau} \frac{\partial \tau}{\partial t}  \\ \frac{\partial x_2}{\partial \tau} \frac{\partial \tau}{\partial t} \end{pmatrix} 

%\begin{equation*}
%	\begin{align}
	%		\dot{x} =  \begin{pmatrix} \dot{x_1} \\ \dot{x_2} \end{pmatrix}  & =  T  \begin{pmatrix}  x_2(t;p) \\ u(t)-p   \end{pmatrix} = \begin{pmatrix}  Tx_2(t;p) \\ T(u(t)-p)   \end{pmatrix} \\
	%		i.e. \ \begin{pmatrix} \dot{x_1} \\ \dot{x_2} \end{pmatrix} &= \begin{pmatrix} \frac{\partial x_1}{\partial t} \\ \frac{\partial x_2}{\partial t} \end{pmatrix} = 
	%		\begin{pmatrix} \frac{\partial x_1}{\partial \tau} \frac{\partial \tau}{\partial t}  \\ \frac{\partial x_2}{\partial \tau} \frac{\partial \tau}{\partial t} \end{pmatrix} 
	%	\end{align}
%\end{equation*}

   %

\chapter{The Classical and Training Approach}

The paper on hand focuses on using quasi-Newton and multi-shooting method for the Training Approach. In this chapter, we shortly introduce the Classical Approach first and then we discuss the Training Approach in greater detail. In the next chapter, we can introduce the quasi-Newton and multi-shooting method, and elaborate in detail how they can be used for the Training Approach. 



\section{The Classical Approach}

As stated in the introduction part, the classical approach is consistent with the $minmax$ approach, during which, two level optimization problems are solved. 

In the lower level, we solve an optimization problem ($max \  f(x,p)$) with respect to $p$, and in the upper level, we continue to find the best solution with respect to $x$, as shown in \ref{minmax}. In the case of the rocket car, the classical approach will be expressed in the following form

\begin{subequations}
	\begin{align}
		\underset{T, u(\cdot)}{min} \  \underset{ p \in \Omega_P, x(\cdot,p)}{max}  \ \   & \  T \\ 
		s.t.  & \ \ x = (x_1, x_2)   \label{ca_rc_x} \\ 
		& \ \  \dot{x} = T  \begin{pmatrix}  x_2(t;p) \\ u(t)-p   \end{pmatrix}, & \ t \in [0,1],  \label{ca_rc_partial} \\
		& \ \ x(0,p) = 0, \label{ca_rc_t0}\\
		& \ \ x_1(1;p) \geq 10, & \ for \ all \ p \in \Omega_P, \label{ca_rc_x1_t1} \\
		& \ \ x_2(t;p) \leq 4, & t \in [0,1], \ for \ all \ p \in \Omega_P,  \label{ca_rc_x2_tc} \\
		& \ \ x_2(1;p) \leq 0, & \ for \ all \ p \in \Omega_P, \label{ca_rc_x2_t1}  \\
		& \ \ T \geq 0, \\
		& \ \ u(t) \in [-10, 10], & t \in [0,1]. 
	\end{align}
	\label{ca_rc}
\end{subequations}

In the Classical Approach, the set of feasible controllable parameters and control functions are given by those $T$ and $u(\cdot)$, which yield feasible trajectories $x(\cdot, p)$ for all $p \in \Omega_P$. The value of the objective function in the lower level does not depend on $p$ and $x(\cdot, p)$. In other words, in this approach, the driver has no prior knowledge about the value of the parameter $p$ and gets no feedback during the process and has to set up the driving strategy in advance. 

\section{The Training Approach}
Contrast to the Classical Approach, in the Training Approach it is assumed that the driver of the rocket car is able to perform optmially for every $p$ because of a preceding training period. Thus the worst possible optimal performance is given by a solution of the problem
\begin{subequations}
	\begin{align}
	   \underset{p \in \Omega_P, T, u(\cdot), x(\cdot,p)}{max}  \ 	\underset{}{min} \   & \  T \\ 
		s.t.  & \ \ x = (x_1, x_2)   \label{ta_rc_x} \\ 
             & \ \  \dot{x} = T  \begin{pmatrix}  x_2(t;p) \\ u(t)-p   \end{pmatrix}, & \ t \in [0,1],  \label{ta_rc_partial} \\
& \ \ x(0,p) = 0, \label{ta_rc_t0}\\
& \ \ x_1(1;p) \geq 10, \label{ta_rc_x1_t1} \\
& \ \ x_2(t;p) \leq 4, & t \in [0,1], \label{ta_rc_x2_tc} \\
& \ \ x_2(1;p) \leq 0, \label{ta_rc_x2_t1}  \\
& \ \ T \geq 0, \\
& \ \ u(t) \in [-10, 10], & t \in [0,1]. 
	\end{align}
	\label{TA_rc}
\end{subequations}

The solution of the Training Approach in paper \cite{MatSch22} is given by a gradient-free method, more precisely, a so-called model-based DFO approach for box-constrained optimization problems is used. The BOBYQA algorithm is chosen for such approach to solve problems of the form
\begin{equation}
	\begin{aligned}
		\underset{x \in \mathcal{R}^n}{min} & \  F(x)  \\ 
		s.t.  & \ a_i \leq x_i \leq b_i, i = 1, ..., n \\
	\end{aligned}
	\label{DFO_bc}
\end{equation}

The name BOBYQA is an acronym for "Bound Optimization BY Quadratic Approximation", and is used to solve lower level problem of \ref{TA_rc}. In the general DFO method, the objective function $F(\cdot)$ is considered a black box. For a given $p$, the parametric lower level OCP is solved with a direct approach and the resulting (finite dimensional) solution is viewed as dependent variable. Furthermore, the uncentainty set $\Omega_P$ is box-shaped, and hence the BOBYQA algorithm is applicable to the problem in the Training Approach.The BOBYQA algorithm has been introduced in details in the paper \cite{MicPow22}, and we reiterate the main idea in the text that follows.  

The method of BOBYQA is iterative, $k$ and $n$ being reserved for the iteration number and the number of variables, respectively. Further, $m$ is reserved for the number of interpolation conditions that are imposed on a quadratic approximation $Q_k(x), x \in \mathcal{R}$, to $F(x), \mathcal{R}$, with $m$ is a chosen constant  integer from the interval $[n+2, \frac{1}{2}(n+1)(n+2)]$. The approximation is available at the beginning of the $k$-th iteration, the interpolation equations have the form
\begin{equation}
  Q_k(y_j)= F(y_j),\   j = 1, 2, ..., m, 
\end{equation}
We let $x_k$ be the point in the set $\{y_j : j = 1, 2, ... , m\}$ that has the property
\begin{equation}
	F(x_k)= min\ \{F(y_j), \  j = 1, 2, ..., m\}, 
\end{equation}
with any ties being broken by giving priority to an earlier evaluation of the least function value $F(x_k)$. A positive number $\Delta_k$, called the “trust region radius”, is also available at the beginning of the $k$-th iteration. If a termination conditions is satisfied, then the iteration stops. Otherwise, a step $d_k$ from $x_k$ is constructed such 
that $ \Vert d_k \Vert \leq \Delta_k $ holds, such that $x = x_k+d_k$ is within the bounds \ref{DFO_bc}, and such that $x_k+d_k$ is not one of the interpolation points $y_j : j = 1, 2, ... , m$. Then the new function value $F(x_k+d_k)$ is calculated, and one of the interpolation points, $y_t$ say, is replaced by $x_k+d_k$, where $y_t$ is different from $x_k$. It follows that $x_{k+1}$ is defined by the formula
\begin{equation}
	x_{k+1}
	\begin{cases}
		 x_k, & F(x_k+d_k) \geq F(x_k) \\
		x_k+d_k  , & F(x_k+d_k) < F(x_k) 
	\end{cases}
\end{equation}

Further, $\Delta_{k+1}$ and $Q_{k+1}$ are generated for the next iteration, $Q_{k+1}$ being subject to the constraints 
\begin{equation}
	Q_{k+1}(\hat{y}_j)= F(\hat{y}_j), \  j = 1, 2, ..., m, 
\end{equation}
at the new interpolation points
\begin{equation}
	\hat{y}_j =
	\begin{cases}
		y_j, & j \neq t, \\
		x_k+d_k  , & j =t 
	\end{cases},  \  j = 1, 2, ..., m.
\end{equation}


%In the method of BOBYQA algorithm, in each iteration $k$ the objective function is approximated by a sequence of quadratic function $Q_k(\cdot)$, such that 
%\begin{equation}
%	\begin{aligned}
%Q_k(z^{k,i}) = F(z{k,i}), \   i = 1, ..., m,
%	\end{aligned}
%	\label{BOBYQA}
%\end{equation}
%for interpolation points $z^{k,i} \in \mathcal{R}$. The number of interpolation points $m$ is constant. Let $x_k \in \ argmin \ \{Q_k(z^{k,i}) | i=1, ...,m\}$. In every iteration $k$, by means of quadrative model, one computes a feasible step $d_k$, which is inside a "trust-region radius" $\Delta_k$, i.e. $\l d_k \leq \Delta_k$. Subsequently, the function $F(\cdot)$ is evaluated at $x_k + d_k$, one interpolation point $z^{k,i}$ is replaced by $x_k + d_k$, and the quadrative model is updated. The sequence $x_k$ is expected to approach a solution of Problem \ref{DFO_bc}.

 











%can assume that $p^0$ is a fixed value in the feasible uncertainty set $\Omega_P$

%The classical approach, aka, the robust optimization  is concerned with optimization problems which involes uncertainy parameters whose value is a priori unknown. 


% approach and the
%

\chapter{The Introduction to Multiple Shooting and Quasi-Newton Method}

In this chapter, we first introduce the basics of solutions to nonlinear problems and the focus on the quasi-Newton and multiple shooting method. 

As stated in the introduction chapter \ref{Chapter1}, a general optimization problem is typically of the form 
\begin{equation}
	\begin{aligned}
		\  \  \ & min \  f(x) \\
		s.t.\ \  & x \in \Omega
	\end{aligned}
	\label{GeneralOpt}
\end{equation}

Here $x \in \Omega$ represents the constraints for which $x$ must satisfy, it may be in the form of $ g(x) = 0,  h(x)  \geq  0$ as in the problem \ref{GeneralMin} in the Introduction chapter, i.e. the feasible set $\Omega = \underset{x}{arg} \ \{ g(x) = 0,  h(x)  \geq  0 \}$. 

Such problem can be solved via Newton's method, which attempts to solve this problem by constructing a sequence $\{x_K\}$ from an initial guess (starting point) $x_0 \in \Omega$ that converges towards a minimizer $x^\star$ of $f(x)$  by using a sequence of first and/or second order Taylor approximations of $f(x)$ around the iterates. The second-order Taylor expansion of $f(x)$ around $x_k$ is
%\begin{multline*}
\begin{align*}
f(x_k + \delta_x) \approx f(x_k) + f'(x_k)\delta x +\frac{1}{2}f''(x_k)\delta_x^2
\end{align*}
where $\delta_x$ represents a small change (with respect to $x$), and $f', f''$ are the first and second order derivatives of the original function $f(x)$. 
%\end{multline*}




%{\displaystyle f(x_{k}+t)\approx f(x_{k})+f'(x_{k})t+{\frac {1}{2}}f''(x_{k})t^{2}.}{\displaystyle f(x_{k}+t)\approx f(x_{k})+f'(x_{k})t+{\frac {1}{2}}f''(x_{k})t^{2}.}
%The next iterate {\displaystyle x_{k+1}}x_{k+1} is defined so as to minimize this quadratic approximation in {\displaystyle t}t, and setting {\displaystyle x_{k+1}=x_{k}+t}{\displaystyle x_{k+1}=x_{k}+t}. If the second derivative is positive, the quadratic approximation is a convex function of {\displaystyle t}t, and its minimum can be found by setting the derivative to zero. Since

%{\displaystyle \displaystyle 0={\frac {\rm {d}}{{\rm {d}}t}}\left(f(x_{k})+f'(x_{k})t+{\frac {1}{2}}f''(x_{k})t^{2}\right)=f'(x_{k})+f''(x_{k})t,}{\displaystyle \displaystyle 0={\frac {\rm {d}}{{\rm {d}}t}}\left(f(x_{k})+f'(x_{k})t+{\frac {1}{2}}f''(x_{k})t^{2}\right)=f'(x_{k})+f''(x_{k})t,}
%the minimum is achieved for

%{\displaystyle t=-{\frac {f'(x_{k})}{f''(x_{k})}}.}{\displaystyle t=-{\frac {f'(x_{k})}{f''(x_{k})}}.}
%Putting everything together, Newton's method performs the iteration

%{\displaystyle x_{k+1}=x_{k}+t=x_{k}-{\frac {f'(x_{k})}{f''(x_{k})}}.}{\displaystyle x_{k+1}=x_{k}+t=x_{k}-{\frac {f'(x_{k})}{f''(x_{k})}}.}

%Here, the function $f(x)$ is typically a non-linear function. 




%Without attempting completeness, we pr















%can assume that $p^0$ is a fixed value in the feasible uncertainty set $\Omega_P$

%The classical approach, aka, the robust optimization  is concerned with optimization problems which involes uncertainy parameters whose value is a priori unknown. 


% approach and the
%\chapter{Rocket Car Case}
Before we dive into the mathematics, we first re-introduce the physics of the rocket car case so that we can model the problem more precisely. Within the training approach, with a given $p$, we want to solve the lower level problem of the following form \ref{TA_lower}
\begin{subequations}
	\begin{align}
    	\underset{}{min} \   & \  T \\ 
		s.t.  & \ \ x = (x_1, x_2)   \label{ta_rc_x} \\ 
		& \ \  \dot{x} = T  \begin{pmatrix}  x_2(t;p) \\ u(t)-p   \end{pmatrix}, & \ t \in [0,1],  \label{ta_rc_partial2} \\
		& \ \ x(0,p) = 0, \label{ta_rc_t2}\\
		& \ \ x_1(1;p) \geq 10, \label{ta_rc_x1_t2} \\
		& \ \ x_2(t;p) \leq 4, & t \in [0,1], \label{ta_rc_x2_tc2} \\
		& \ \ x_2(1;p) \leq 0, \label{ta_rc_x2_t1_2}  \\
		& \ \ T \geq 0, \\
		& \ \ u(t) \in [-10, 10], & t \in [0,1], \\
		& \ \ p, \   a \ given \ value
	\end{align}
	\label{TA_lower2}
\end{subequations}

In summary, we want to find the minimum time that the rocket car moves from the starting state (the position is at the origin point, and the speed is zero) to an ending state when the position is at least at point 10 or beyond, and the speed is less than or equal to zero (a negative speed indicates the car is moving in a direction that decreases the position), with constraints on the acceleration/deceleration value and the speed during the whole process. 

Because our objective is to minimize the time between starting state and ending state, i.e. the variable $T$, which is unknown, we cannot define a time horizon over which we will discretize and optimize. Therefore, a new variable $t$ is defined as follows: 
\begin{equation*}
 	t= \frac{\tau}{T} \in [0,1] \quad \tau \in [0, T]
 	\label{eqn:timet}
\end{equation*}

Where $\tau$ is the real time between starting time $0$ and ending time $T$, and $t$ is the relative time between $0$ and $1$.  The equation \ref{ta_rc_partial2} can be also written as 

\begin{subequations}
		\begin{align}
 \dot{x} =  \begin{pmatrix} \dot{x_1} \\ \dot{x_2} \end{pmatrix}  & =  T  \begin{pmatrix}  x_2(t;p) \\ u(t)-p   \end{pmatrix} = \begin{pmatrix}  Tx_2(t;p) \\ T(u(t)-p)   \end{pmatrix} \label{eq_difT} \\ 
 \begin{pmatrix} \dot{x_1} \\ \dot{x_2} \end{pmatrix} &= \begin{pmatrix} \frac{\partial x_1}{\partial t} \\ \frac{\partial x_2}{\partial t} \end{pmatrix} = \begin{pmatrix} \frac{\partial x_1}{\partial \tau} \frac{\partial \tau}{\partial t} \\ \frac{\partial x_2}{\partial \tau} \frac{\partial \tau}{\partial t} \end{pmatrix} =  \begin{pmatrix} \frac{\partial x_1}{\partial \tau} T \\ \frac{\partial x_2}{\partial \tau}T \end{pmatrix} =     \begin{pmatrix}  Tx_2(t;p) \\ T(u(t)-p)   \end{pmatrix} \\
 \begin{pmatrix} \frac{\partial x_1}{\partial \tau}  \\ \frac{\partial x_2}{\partial \tau} \end{pmatrix} & =     \begin{pmatrix}  x_2(t;p) \\ (u(t)-p)   \end{pmatrix} \label{eq_difTau}
 	\end{align}
 \end{subequations}

In summary, the equation $\frac{\partial x_1}{\partial \tau}= x_2(t;p) $ means the change in the position in real time is proportional to the speed at that moment. And the equation $\frac{\partial x_2}{\partial \tau} = u(t)-p $ means the change in speed is proportional to the acceleration/deceleration value at that moment. 

To use multiple shooting and quasi-Newton method, we discretize the interval $t\in [0,1]$ into subinterval $[t_{k-1}, t_k], k = 1, 2, ..., m$, where $0 =t_0, t_1, t_2, ...,t_{k-1}, t_k, ..., t_m=1$. We solve the OCP within each interval, and enforce the matching condition at the boundary of each interval. 

The equation \ref{eq_difT} is equivalent to \ref{eq_difTau}





% t&= \frac{\tau}{T} \in [0,1] \quad \tau \in [0, T]
%\\
%i.e. \ \begin{pmatrix} \dot{x_1} \\ \dot{x_2} \end{pmatrix} &= \begin{pmatrix} \frac{\partial x_1}{\partial t} \\ \frac{\partial x_2}{\partial t} \end{pmatrix} = 
%\begin{pmatrix} \frac{\partial x_1}{\partial \tau} \frac{\partial \tau}{\partial t}  \\ \frac{\partial x_2}{\partial \tau} \frac{\partial \tau}{\partial t} \end{pmatrix} 

%\begin{equation*}
%	\begin{align}
%		\dot{x} =  \begin{pmatrix} \dot{x_1} \\ \dot{x_2} \end{pmatrix}  & =  T  \begin{pmatrix}  x_2(t;p) \\ u(t)-p   \end{pmatrix} = \begin{pmatrix}  Tx_2(t;p) \\ T(u(t)-p)   \end{pmatrix} \\
%		i.e. \ \begin{pmatrix} \dot{x_1} \\ \dot{x_2} \end{pmatrix} &= \begin{pmatrix} \frac{\partial x_1}{\partial t} \\ \frac{\partial x_2}{\partial t} \end{pmatrix} = 
%		\begin{pmatrix} \frac{\partial x_1}{\partial \tau} \frac{\partial \tau}{\partial t}  \\ \frac{\partial x_2}{\partial \tau} \frac{\partial \tau}{\partial t} \end{pmatrix} 
%	\end{align}
%\end{equation*}



 

%
% 

\chapter{Rocket car case}
Since the approaches we are going to use in this paper will be demonstrated with the case of rocket car, we decide to describe the rocket car case first. So that, when we are discussing our approaches, we can directly describe how they can be used in solving the rocket car case. The description of the rocket car case is mostly coming from the paper \cite{MatSch22}, with content either verbatim or in a modified form. 

We consider the rocket car case with state constraints, i.e. the one-dimensional movement of a mass point under the influence of some constant acceleration/deceleration, e.g. modeling head-wind or sliding friction, which can accelerate and decelerate in order to reach a desired position. The mass of the car is normalized to 1 unit and the constant acceleration/deceleration enters the model in form of an unknown parameter $p \in \Omega_P \subset \mathbb{R}$ suffering from uncertainty, with the uncertainty set $\Omega_P$ convex and compact. We consider a problem in which the rocket car shall reach a final feasible position and velocity in a minimum time: 


\begin{equation}
	\begin{aligned}
		\underset{T, \mu(\cdot), x(:,p)}{min} \   & \  T \\ 
		s.t.  & \ \ x = (x_1, x_2)  \\ 
		& \ \  x = T  \begin{pmatrix}  x_2(t;p) \\ u(t-p)   \end{pmatrix}, \ t \in [0,1]   \\
		& \ \ x(0,p) = 0,\\
		& \ \ x_2(4;p) \leq 4, t \in [0,1] \\	
	 \end{aligned}
\end{equation}



\begin{equation}
	% \begin{aligned}
		\underset{T, \mu(\cdot), x(:,p)}{min} \   & \  T \\ 
		\begin{subequations}
				\begin{alignat}{2}
		s.t.  & \ \ x = (x_1, x_2)   \label{sub-1:1} \\ 
		      & \ \  x = T  \begin{pmatrix}  x_2(t;p) \\ u(t-p)   \end{pmatrix}, \ t \in [0,1]  \label{sub-2:1} \\
		      & \ \ x(0,p) = 0, \label{sub-2:3}\\
		      & \ \ x_2(4;p) \leq 4, t \in [0,1] \label{sub-2:4}
	\end{alignat}
     \end{subequations}	
	% \end{aligned}
\end{equation}


%\begin{equation}
%		IP : \  \  \  max \sum_{e \in E} w_ex_e, \label{eq:1} \\
%	\begin{aligned}
%		s.t.  \  \  \  &\sum_{e \in E_i} x_e \leq 1 \  for \ all \ i \in V,  \\
%		& x_e \in \{0, 1\} \ for \ all \ e \in E\\
%	\end{aligned}
%\end{equation}


\begin{subequations}\label{eq:1}
	\begin{alignat}{2}
		m_{i}' &=0   & m_{ij} &= m_{i} \label{sub-eq-1:1}\\
		mx_{j}' &=0  & m_{ij} &= -\frac{1}{\lambda_j\cos\alpha_j}\cdot mx_j \label{sub-eq-2:1} \\
		my_{j}' &= 0 & m_{ij} &= -\frac{1}{\lambda_j\sin\alpha_j}\cdot my_j \label{sub-eq-3:1} \\
		J_{j}' &= 0  & m_{ij} &= -\frac{1}{\lambda_j^2} \cdot J_j \label{sub-eq-4:1} \\
		m_{j}' &= 0  &\quad\Longrightarrow\quad m_{ij} &= -m_j \tag{2}\label{eq:2}
	\end{alignat}
\end{subequations}


%{\let\clearpage\relax \chapter{bar}}
%% Put your contents here


%\usepackage{amsthm}
%\usepackage{amssymb}

%\usepackage[normalem]{ulem}
%\usepackage{amssymb}
%\usepackage{amsfonts}
%\usepackage[normalem]{ulem}
%\usepackage                             {amsmath}
%\usepackage                             {graphicx}


%\usepackage{etoolbox}
%\makeatletter
%\patchcmd{\chapter}{\if@openright\cleardoublepage\else\clearpage\fi}{}{}{}
%\makeatother
%\usepackage{mathrsfs}

%\usepackage                             {mathrsfs}
%\usepackage{dsfont}
%\usepackage{amsmath}
%% \usepackage[thmmarks,amsmath]{ntheorem}
%\usepackage{amsmath}
%\usepackage{amsthm}
%\usepackage[thmmarks,amsmath]{ntheorem}

%\newcommand{\attn}[1]{\textbf{#1}}
%\theoremstyle{definition}

%\newtheorem{example}{Example}
%\newtheorem*{note}{Note}

%\newcommand{\P}{\mathbb{P}}
% links
% used pagages
%\usepackage     [utf8]                  {inputenc}

% definition given in equation \ref{eq_PW_obj}.
%% within subinterval $\mathbb{I}_j$ with the solution $x(\tau; x_{j-1}, u_{j-1})$ Within subinterval $\mathbb{I}_j$, with initial guess $X_j = (x_{j-1}, u_{j-1})$,  


% discretize the time $t \in [t_0, t_f]$ and the original problem is split into multiple subintervals, with the constraints discretized and applied to each subinterval. We also need to introduce the initial guess for each subinterval and add the matching condition at each boundary. In the end, we turn the original problem in the form of equation  \ref{P2_OPM} into piecewise OCPs with augmented parameters and constraints. 
%In the numerical implementation for our chosen case, the Runge–Kutta RK4 method will also be used.  


%& w = (x_0, u_0, x_1, u_1, ..., x_{m-1}, u_{m-1}) \\
%     &   t_0 = \tau_0 < \tau_1 < ... <\tau_j < ... < \tau_m = t_f 
%, \ \ j = 1, 2, ...,m 

%			s.t.\ \ &  \dot{x} (t) = f(x(t), u(t)), \ \ (system \ dynamics)   \label{P2_sd} \\
%& g(x(t), u(t)) = 0, \ \  (path\  equality\  constraints)  \label{P2_ec}\\
%& h(x(t), u(t)) \leq 0, \ \ (path\  inequality \ constraints)  \label{P2_inc}\\
%& x(t_0) = x_0, \ \ (initial \ value) \\
%& r(x(t_f)) \leq 0, \ \ (terminal \ constraints)  \label{P2_final} \\
%& x^{lower} \leq x(t) \leq x^{upper}   \label{P2_box_x} \\ 
%& u^{lower} \leq u(t) \leq u^{upper}   \label{P2_box_u} \\ 
%& t \in [t_0, t_f]

%where $  (X_j,  x(\tau_j; x_{j-1}, u_{j-1}))  \in \Theta_j $ represents that the initial guess $x_{j-1}, u_{j-1}$  and the solution $x(\tau_j; x_{j-1}, u_{j-1})$  for subinterval $\mathbb{I}_j = [\tau_{j-1}, \tau_j]$ satisfy the constraints  \ref{P2_ec}, \ref{P2_inc}, \ref{P2_final}, \ref{P2_box_x}, and \ref{P2_box_u} within subinterval $\mathbb{I}_j$. The solution $x(\tau_j; x_{j-1}, u_{j-1})$ comes from solving the dynamic system in equation \ref{P2_sd} for the subinterval $\mathbb{I}_j$ with numerical method used. The numerical method can be one of the Runge–Kutta methods as discussed before. The objective function $F_j$ of subinterval is calculated as followin


%is a numerical method to solve ordinary differential equations derived from the trapezoidal rule for computing integrals. The
%\begin{equation}
%	\int_{t_k}^{t_{k+1}} x_2(t;p) d \tau  \approx \frac{x_2(t_{k+1}) + x_2(t_k)}{2} (\tau_{k+1} -\tau_k)
%	\label{mid_approx}
%\end{equation}


%There are many methods that can be used to approximate the integrals, one of the widely used method is the Runge–Kutta method. With initial value $x_1(t_k),x_2(t_k), u(t_k)$ given at each subinterval $[t_{k}, t_{k+1}]$, based on the differential equation \ref{ta_rc_partial2}, we can find the (approximation) solution within this subinterval. 
% However, in order to get the feasible optimal solution for the original problem (i.e. the whole interval), we need to enforce the matching condition at the boundary of each subinterval as well as the constraints defined in equations \ref{ta_rc_t2}-\ref{ta_ut}.  These constraints can be addressed with the KKT method discussed in the next Section \ref{Sec_KKT}.



%	\begin{subequations}
	%	\begin{align}
		%		x_1(t_{k+1}) -  x_1(t_k)  &= \int_{t_k}^{t_{k+1}} x_2(t;p) d \tau \label{eq_x1_int} \\
		%		x_2(t_{k+1}) -  x_2(t_k)  &= \int_{t_k}^{t_{k+1}} (u(t)-p) d \tau \\
		%		t \in [t_{k}, t_{k+1}], \  k &= 0, 1, 2, ..., m-1, \  t_0 =0, t_m =1
		%	\end{align}
	%\end{subequations}
	
	
	%\begin{equation}
	%	\int_{t_k}^{t_{k+1}} x_2(t;p) d \tau  \approx \frac{x_2(t_{k+1}) + x_2(t_k)}{2} (\tau_{k+1} -\tau_k)
	%	\label{mid_approx}
	%\end{equation}
	% To solve the IVP, the natural idea to i The solution can be found either numerically or analytically, if analytical solutions exist. Nevertheless,  approximately by nume 
	%When we $t \in [t_0, t_f]$
	%But the piecewise OCPs can be aggregated together due to their non-overlapping properties and the same structure, i.e. they can be aggregated to one objective function with the constraints expressed in matrix form. The transformed problem can then be solved with KKT approach and quasi Newton method. In the next two sections, we explain in detail the KKT condition and quasi Netwon method respectively.
	
 
%Within each discretized subinterval, we form a solution and ensure the continuity for the whole interval by enforcing the matching condition at the boundary of each suninterval. With the same idea, we can use mutiple shooting to solve optimal control problems. For the rocket car problem \ref{TA_rc}
%, a constant control in each subinterval does not affect our solulation if we  which is easy to implement for our chosen case. 	

	%Define total cost function for all the errors arised from the solutions of the previous step. We use the notation $F_j$ as the cost function for the subinterval $\mathbb{I}_j$, with the sum of $F_j$ over all the subintervals as the total cost function. 
	
		% $$ based on some updating  \ref{P2_ec}, \ref{P2_inc}, \ref{P2_final}, \ref{P2_box_x}, and \ref{P2_box_u}, which are applicable in subinterval $\mathbb{I}_j$. %. We use the notation $\Theta_j$ as the collections of contrstraints
		
%where $w = (x_0, u_0, x_1, u_1, ..., x_{m-1}, u_{m-1}, u_m) $, and $x_{j-1}, u_{j-1}$ is the initial guess  for interval $\mathbb{I}_j = [\tau_{j-1}, \tau_j]$, 

%	\begin{subequations}
	%	\begin{align}
		%		\underset{x(\cdot), u(\cdot)}{\text{min}}   \ &  F(x(\cdot), u(\cdot))  = \int_{t_0}^{t_f}L(x(t), u(t))dt + E (x(t_f)) \label{P2_cost} \\
		%		s.t.\ \ &  \dot{x} (t) = f(x(t), u(t)), \ \ (system \ dynamics)   \label{P2_sd} \\
		%		& g(x(t)) = 0 \  or \leq 0, \ t \in [t_0, t_f]\  (path\  equality\ or\ inequality\ constraints)  \label{P2_ec}\\
		%		& h(x(t), u(t)) =0\  or  \leq 0,\ t \in [t_0, t_f] \ (mixed \ control-state  \ constraints)  \label{P2_inc}\\
		%		& x(t_0) = x_0, \ \ (initial \ value) \\
		%		& r(x(t_f)) \leq 0, \ \ (terminal \ constraints)  \label{P2_final} \\
		%		& u^{lower} \leq u(t) \leq u^{upper}   \label{P2_box_u} \\ 
		%		& t \in [t_0, t_f]
		%	\end{align}
	%	\label{P2_OPM}
	% \end{subequations} 

%In order to get the feasible solution, we can take the constraints into consideration by Lagrange multipliers with the Karush–Kuhn–Tucker (KKT) condition incorporated. 

 % is one of the widespread and sucessfully used techniques to address the constraints of NLP. 		