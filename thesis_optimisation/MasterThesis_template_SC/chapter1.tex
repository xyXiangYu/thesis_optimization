

\section{Introduction}

Many real life problems, can be modeled as parameterized optimization problems, such as the therapy design of Cerebral Palsy problem described in \cite{MatSch22}. In this paper, we focus on the using gradient method to solve parameterized optimization problems, with a case study in state constrained rocket car. 

Without giving a rigorous condition and definition,  a general optimization problem is typically of the form

\begin{equation}
	\begin{aligned}
		 \  \  \ & min \  f(x) \\
		s.t.  \  \  \ & g(x) = 0, \\ 
		              &  h(x)  \geq  0 
	\end{aligned}
\end{equation}

where $f(x)$ is the objective or cost function, and $g(x) = 0$ and $h(x)  \geq  0$ are the constraints. Some optimization problems may have uncertain parameters whose value are priori unknown, and the optimal objective value depends on the parameter value. This kind of problem is called the parameterized optimization problems and is of the form 


\begin{equation}
	\begin{aligned}
		\  \  \ & min \  f(x, p) \\
		s.t.  \  \  \ & g(x, p) = 0, \\ 
		&  h(x,p)  \geq  0  \\ 
		& x = x(p) \\
		& x = x(p^0) \  if \  p = p^0 \\
		& p \in \mathbb{P}		
	\end{aligned}
\end{equation}
where $p^0$ is a fixed value in the feasible uncertainty set $\mathbb{P}$, where the parameter $p$ can take value from.

In the paper \cite{MatSch22}, multiple methods of solving the parameterized optimization problem have been discussed, and the main focus (of solving the  Cerebral Palsy problem) in that paper is the "worst-case treatment planning by bilevel optimal control". Assuming that $\tilde{p}$ lies in an uncertainty set $\mathbb{P}$, our objective is to identify a worst possible solution with respect to $\tilde{p}$, i.e. solving a lower level problem. Based on the result of lower level, we continue to find the best solution with respect to $x$, i.e. solving a upper level problem. The "worst-case treatment planning by bilevel optimal control", i.e. a  bilevel optmization problem, is an optmization problem in which an optimization problems enters the constraints. 
Mathematically, the problem can be formulated as following


\begin{equation}
	\begin{aligned}
		\  \  \ &  \underset{x}{min} \  \tilde f(x) \\
		where  \  \  \ & \tilde f(x) =    \begin{cases}
		  	\underset{p \in \mathbb{P}}{max} & \ f(x,p) \\
			s.t.   & \  g(x, p) = 0, \  h(x,p)  \geq  0 \\
		\end{cases}  
	%	&  h(x,p)  \geq  0  \\ 
	%	& x = x(p) \\
%		& x = x(p^0) \  if \  p = p^0 \\
%		& p \in \mathbb{P}		
	\end{aligned}
\end{equation}

In a simplified notation, the problem can be written as 

\begin{equation}
	\begin{aligned}
		\underset{x}{min} \   \underset{p \in \mathbb{P}}{max} & \  f(x,p) \\ 
		s.t.  & \  g(x, p) = 0, \  h(x,p)  \geq  0 \\
	\end{aligned}
\end{equation}
% \underset{\theta \in \mathscr{T}_{t,T}}{\sup} 

% c =  \underset{x \in \mathbf{P}-x^\ast}{inf}\frac{w'(x^\ast -x)}{|x^\ast -x|}$$





which include the nominal solution, the robust optimization and the training approach. 


Using robust optimization, one can robustify or immunize a solution against parameter uncertainty in terms
of feasibility and optimality.






In this problem, the optimal objective value depends on the 







In some problems, uncertain parameters whose value are priori unknown appears in the optimization problems. 



with optimization problems which involve uncertain parameters whose value is a priori unknown. 

Consider the dual problem of the LP, denoted below as DUAL: 








In a commonly applied approach, the human gait is modeled as a solution of a multistage OCP with a predefined order of model phases and jumps in the differential
states at phase transition, cf., e. g., [71, 105]. The model phases correspond to the
occurring foot-ground contact configurations during a gait cycle. However, as explained previously, it is observed that the order and number of model phases can
change due to medical treatment. Thus, for a predictive modeling of intervention
outcomes, such a model is only useful to a limited extend. One way to overcome this
shortcoming is to model the human gait as the solution of an OCP that is governed by
a switched dynamical system in which the number or order of model phases is free
and subject to optimization. In [26] the authors investigate switched OCPs, however without considering jumps in the differential states at switching. We extend
the framework presented in the latter reference to make it suitable for our purposes.
In doing so, we extend the Partial Outer Convexification approach [127]. Switching indicators and switching costs play a crucial role in our approach. Altogether,
we present a novel approach for the numerical solution of switched systems with
switching costs and possible jumps in the differential states at phase transitions. Details are given in Chapter 5.





In computational finance, option pricing begins with an assumption that the underlying asset follows some stochastic process. Similarly, physical phenomena can be modeled with the assumption that the underlying is a stochastic process. That is why many models used in computational finance and physics are interchangeable. The canonical Black–Scholes–Merton (BSM) model is essentially the same as heat equation in Physics. They have the same partial differential equation and the same stochastic process. The market price in computational finance and the atoms movement in physics essentially follows the same kid of random walk. 


In this paper, we focus on pricing of American options, i.e. essentially solving partial differential equations with boundary conditions. Two main methods have been discussed, the first one is finite difference, with different optimization method used, and the second is COS method proposed by Fang Fang. 


Option pricing comes into existence with the paper by Fischer Black,  Myron Schole and Robert C. Merton in 1973. In the BSM model, the asset price $S_t$, at time $t$, is assumed to satisfy the following stochastic differential equations, knowns as the BSM SDE: 
\begin{equation} 
	dS_t= \mu S_tdt + \sigma S_t dW_t
	\label{BSModel}
\end{equation} 
where usually $\mu = r -q$ and $r$, $q$ , $\sigma$ are a continuous risk-free interest rate, a continuous dividend rate and the instantaneous volatility respectively. This equation models the asset’s log returns as growing at a constant rate of $r - q$ and having a volatility of $\sigma$. The solution for the stochastic differential equation above is as follows, 