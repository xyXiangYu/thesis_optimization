

\chapter{Rocket car case}
Since the approaches we are going to use in this paper will be demonstrated with the case of rocket car, we decide to describe the rocket car case first. So that, when we are discussing our approaches, we can directly describe how they can be used in solving the rocket car case. The description of the rocket car case is mostly coming from the paper \cite{MatSch22}, with content either verbatim or in a modified form. 

We consider the rocket car case with state constraints, i.e. the one-dimensional movement of a mass point under the influence of some constant acceleration/deceleration, e.g. modeling head-wind or sliding friction, which can accelerate and decelerate in order to reach a desired position. The mass of the car is normalized to 1 unit\footnote{We do not specify the unit on purpose since the actual unit, either one kilogram or meter, does not play a role in the modeling. We are more concerned about the scale.} and the constant acceleration/deceleration enters the model in form of an unknown parameter $p \in \Omega_P \subset \mathbb{R}$ suffering from uncertainty, with the uncertainty set $\Omega_P$ convex and compact. We consider a problem in which the rocket car shall reach a final feasible position and velocity in a minimum time: 



\begin{subequations}
	\begin{align}
		\underset{T, u(\cdot), x(:,p)}{min} \   & \  T \\ 
		s.t.  & \ \ x = (x_1, x_2)   \label{rc_x} \\ 
		& \ \  \dot{x} = T  \begin{pmatrix}  x_2(t;p) \\ u(t)-p   \end{pmatrix}, & \ t \in [0,1],  \label{rc_partial} \\
		& \ \ x(0,p) = 0, \label{rc_t0}\\
		& \ \ x_1(1;p) \geq 10, \label{rc_x1_t1} \\
		& \ \ x_2(t;p) \leq 4, & t \in [0,1], \label{rc_x2_tc} \\
		& \ \ x_2(1;p) \leq 0, \label{rc_x2_t1}  \\
		& \ \ T \geq 0, \\
		& \ \ u(t) \in [-10, 10], & t \in [0,1]. 
	\end{align}
\label{rc}
\end{subequations}

where $x$ represents the variables of the rocket car, and it has two components $ x = (x_1, x_2)$. The first component $x_1$ is the (time-transformed) position of the rocket car. The second component $x_2$ is (time-transformed) velocity of the rocket car. The condition \ref{rc_t0}, i.e. $x(0,p) = 0$, indicates that at $t=0$, both the position and velocity of the car is $0$. The condition \ref{rc_x1_t1}, i.e. $x_1(1;p) \geq 10$, indicates that the position of the car at $t=1$ must be greater or equal to $10$. The condition \ref{rc_x2_t1}, i.e. $x_2(t;p) \leq 4$, indicates that the velocity of the car is always smaller or equal to 4 across the whole period. The condition \ref{rc_x2_t1}, i.e. $x_2(1;p) \leq 0$, indicates that the velocity of the car at $t=1$ is always smaller or equal to $0$. Here, a negative velocity means that the car is moving in a direction that decreases the position. 

The decision variable in the problem \ref{rc} is the controllable parameter T, which encodes the process duration of the corresponding problem with free end time. The control function $ u: [0,1] \rightarrow \mathbb{R}$ represents the acceleration/deceleration value, and is dependent on the unknown parameter $p$, as shown in the condition \ref{rc_partial}. The second component of the condition \ref{rc_partial}, i.e. $\dot{x_2} = T (u(t)-p)$, indicates the change in the velocity of the car at time $t$ is subject to the value of $T, u(t)$ and $p$. The first component of the condition \ref{rc_partial}, i.e. $\dot{x_1} = Tx_2(t;p)$, indicates the position of the car at time $t$ is related to the value of $T$ and the velocity $x_2(t;p)$ at time $t$. $x(t:p)$ is a dependent variable, and is uniquely determined by $T, u(\cdot)$ and $p$. The goal is to minimize $T$ such that the variable $x(t:p)$ satisfies all the conditions in \ref{rc}. 






